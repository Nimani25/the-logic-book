\subsection*{منابع}
\begin{LTR}
\begin{enumerate}
  \item Introduction to Lattice and Order, B. A. Davey, H. A. Priestley, 2nd ed., 2002
  \item Lattice Theory, Foundation, G. Gr\~atzer, 2011
  \item Lattice Theory, Garret Birkhoff, 3rd ed., 1967
  \item A Course in Universal Algebra, Stanley Burris, 1981
  \item Universal Algebras, 2nd ed., 1979
  \item Lattices and Ordered Sets, Steven Roman, 2008
  \item Topoi: The Categorical Analysis of Logic, R. Goldblatt, 1984
\end{enumerate}
\end{LTR}

\subsection{سرفصل‌های درس}
\begin{itemize}
  \item مجموعه‌های مرتّب\LTRfootnote{Poset}
  
  تعاریف اوّلیه - مثال - زنجیر و پادزنجیر - نگاشت بین مجموعه‌های مرتّب - دوگان پُست - \lr{‌Bottom - Top} - اعضای مینیمال و ماکسیمال - جمع مجزا - ضرب - جمع خطّی\LTRfootnote{Linear Sum} - مجموعه‌های کاهشی و افزایشی - ویژگی نقطهٔ ثابت.
  \item مشبّکه‌ها - مشبّکه به عنوان یک مجموعهٔ مرتّب - مشبّکه به عنوان یک ساختار جبری - زیر مشبّکه - ضرب و همریختی - ایدئال و فیلتر - شبکه‌های کامل - شرط زنجیر - عناصر $\vee$-تحویل‌پذیر - مشبّکه‌های ماژولار و پخشی - قضیهٔ \lr{$M_3 - N_5$}.
  \item مشبّکه‌های پخشی به همراه $\to$ - جبرهای بولی - جبر هیتینگ - برش مک‌نایل-ددکیند.
  \item نمایش مشبّکه‌ها: قضیهٔ دوگان‌سازی استون-پریستلی-ایساکیا \dots
  \item نظریهٔ رسته‌ها: مقدّمات - توپوس و ارتباط آن با جبرهای هیتینگ.
\end{itemize}

\subsection{تاریخچه}
شروع اوّلیهٔ این نظریه را می‌توان به کارهای دمورگان و بول در حوزهٔ \lr{``Algebra of Logic''} مرتبط کرد که بعدها باعث پیدایش \lr{Lattice Theory} شده است. به ویژه با کتاب ۱۹۳۰ بیرکاف.
\begin{LTR}
  \begin{itemize}
    \item Bool: Mathematical Analysis of Logic
    \item Demorgan: Formal Logic (1847)
  \end{itemize}
\end{LTR}
با کارهای این دو نفر تا نیم‌قرن فقط جبرهای بولی تنها مشبّکه‌هایی بودند که مورد توجّه پژوهشگران بودند. البته این توجّه محدود می‌شد به نقشی که در منطق داشتند. در این مدّت باید به کارهای شرویدر\LTRfootnote{Schr\"oder} اشاره کرد که مفصّل {\red (؟)} موضوع \lr{``Algebra of Logic''} را مطالعه کرد. (۱۸۹۵) به ویژه مطالعهٔ رابطه‌های دوتایی که توسّط پیرس\LTRfootnote{Pierce} شروع شده بود. در واقع در ۱۸۸۰ پیرس تشخیص داد که بین جبرهای بولی و $\to$ ارتباطی هست. او رابطه‌های مرتّب جزیی را تعریف کرد و نشان داد که برای هر دو عنصر $x$ و $y$ از مجموعهٔ مرتّب جزیی $P$، اگر عنصر‌های کوچکترین کران بالا و بزرگترین کران پایین، که آن‌ها را به ترتیب با $x \vee y$ و $x \wedge y$ وجود داشته باشد، آن‌گاه $(P, \wedge, \vee)$ یک مشبّکه است.